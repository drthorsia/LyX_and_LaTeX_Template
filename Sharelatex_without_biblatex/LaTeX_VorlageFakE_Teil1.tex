
\chapter*{Autor(en)}

Maria Mustermann\\
Matrikelnummer 7068602\\
Studiengang: Elektrotechnik im Praxisverbund\\
Studienrichtung: Elektromobilität\\
\vspace{4mm}
\\
Max Musterfrau\\
Matrikelnummer 7068603\\
Studiengang: Elektrotechnik im Praxisverbund\\
Studienrichtung: Elektromobilität

\subsection*{Erstprüferin}

Prof. Dr.-Ing. Marlene Muster\\
( ggf. Institut für ...)\\
Ostfalia Hochschule für Angewandte Wissenschaften -- Hochschule Braunschweig/Wolfenbüttel\\
Salzdahlumer Straße 46/48\\
38302 Wolfenbüttel

\subsection*{Zweitprüfer}

Michael Exampel, M. Eng.\\
XYZ GmbH \& Co. KG.\\
Bahnhofstraße 42\\
32512 Musterdorf

\subsection*{Bearbeitungszeitraum}

Beginn: 2019-02-29, Ende: 2019-09-19

\vfill{}


\subsection*{Erklärung}

-- Hier den Text aus der Prüfungsordnung einfügen, in dem erklärt
wird, dass die Arbeit selbständig erstellt wurde --

\vspace{15mm}
Ort/Datum eigenhändige Unterschrift

\newpage{}

\subsection*{Abstract}

Dies ist das Abstract. Hier steht, worum es geht, welche Methoden
angewendet wurden und was die Ergebnisse sind.

In dieser Vorlage sind viele Verzeichnisse vorhanden. Besprechen Sie
mit allen Beteiligten (Betreuer(in), Prüfer(in), Zweitprüfer(in)),
welche Verzeichnisse in die Arbeit mit aufgenommen werden sollen.

\subsection*{Anmerkungen zu Kooperationen (Nur wenn unbedingt erforderlich)}

Wenn notwendig, kann hier auf eventuelle Kooperationen mit Partnern
hingewiesen werden. Weiterhin ist hier Platz für Danksagungen und
Widmungen.

\tableofcontents{}

\chapter*{Abkürzungsverzeichnis}

\begin{multicols}{2}
\raggedcolumns
\begin{acronym}[aaaaaaaa]
\setlength{\itemsep}{-\parsep}

\acro{adc}[\normalfont{ADC}]{Analog Digital Converter}

\acro{dac}[\normalfont{DAC}]{Digital Analog Converter}

\acro{dms}[\normalfont{DMS}]{Dehnungsmessstreifen}

\acro{hil}[\normalfont{HiL}]{Hardware in the Loop}

\acro{io}[\normalfont{I/O}]{Input/Output}

\acro{lookuptable}[\normalfont{LUT}]{Look Up Table}

\acro{overexcitationlimiter}[\normalfont{OEL}]{Overexcitation Limiter}

\acro{powersystemstabilizer}[\normalfont{PSS}]{Power System Stabilizer}

\acro{rapidcontrolprototyping}[\normalfont{RCP}]{Rapid Control Prototyping}

\acro{statorcurrentlimiter}[\normalfont{SCL}]{Stator Current Limiter}

\acro{underexcitationlimiter}[\normalfont{UEL}]{Underexcitation
Limiter}

\acro{vcl}[\normalfont{VCL}]{VAR Controller Logic}

\end{acronym}
\end{multicols} 

\chapter*{Symbolverzeichnis}

\begin{multicols}{2}
\raggedcolumns 
\begin{acronym}[aaaaaaaa]
\setlength{\itemsep}{-\parsep}

\acro{Di}[$D_{\mathrm{i}}$]{Dämpfung des geschlossenen Stromregelkreises}

\acro{e}[e]{Induzierte Ankerspannung der Gleichstrommaschine}{\acroextra{in $s^{-1}$}}

\acro{Ebase}[$E_{\mathrm{base}}$]{Bezugseffektivwert einer Spannung}

\acro{efd}[$e_{\mathrm{fd}}$]{Normierte Statorspannung der d-Achse}

\acro{EFD}[$E_{\mathrm{FD}}$]{Normierte Feldspannung (Erregersystem)}

\acro{Di}[$e_{\mathrm{fd,base}}$]{Bezugsfeldspannung der d-Achse}

\acro{Di}[$E_{\mathrm{FD,max}}$]{Maximum der normierten Feldspannung
(Erregersystem)}

\end{acronym}
\end{multicols} 

\listoftables

\lstlistoflistings

\listoffigures


\chapter{Einleitung}

\pagenumbering{arabic} Über diese Vorlage: Es handelt sich um eine
Vorlage für wissenschaftliche Arbeiten mit dem Satzssystem LaTeX.
Diese Vorgabe liegt in zwei Formaten vor: LyX und LaTeX. LyX ist ein
freier WYSIWYM (What you see is what you mean)-Editor, der LaTeX-Code
erzeugt, der dann in eine PDF-Datei gewandelt wird. 

Insbesondere bei Gruppenarbeiten bietet es sich alternativ an, als
Editor Overleaf zu nutzen. Es handelt sich um einen kollaborativen
Editor, bei dem in Echtzeit mehrere Personen gleichzeitig ein Textdokument
(LaTeX-Code) bearbeiten können. Als Studierende der Ostfalia Hochschule
können Sie unter \url{www.academiccloud.de} eine Installation von
Overleaf nutzen, deren Server in Deutschland betrieben werden .

Als Schriftart für dieses Dokument wurde STIX (\url{https://www.stixfonts.org/})
gewählt, weil davon auszugehen ist, dass sie sich (Stand Anfang 2021)
in Zukunft in der Wissenschaftsliteratur stark verbreitet wird. Die
Schriftart auf dem Deckblatt ist Nimbus Sans L, weil es eine große
Ähnlichkeit zur Ostfalia-Vorlage für Abschlussarbeiten hat. An dieser
Vorlage wird kontinuierlich verbessert. Bei Fragen oder Feedback dazu
freut sich Prof. Dr. T. Siaenen unter t.siaenen@ostfalia.de über eine
E-Mail.

Es wird hier auch beschrieben, wie eine Qualitätssicherung der Ergebnisse
erfolgt. Bei experimentellen Arbeiten könnte das ein Vergleich zwischen
den experimentellen Ergebnissen und einer Simulation sein. Bei konstruktiv-planerischen
Aufgaben kann eine Qualitätssicherung durch eine Überführung der Aufgabenstellung
ein ein standardisiertes Problemlösungsmuster erfolgen. Beispiel:
Die Aufgabe besteht darin, aus vielen Produkten das ,,beste`` auszuwählen.
Dann wird die Qualität der Arbeit dadurch sichergestellt, dass eine
Nutzwertanalyse durchgeführt wird.

\chapter{Gliederung: Kapitel}

\label{chap:markeeineskapitels}Wichtig bei der Gliederung eines Dokumentes:
Eine Untergliederung hat immer mindestens zwei Einträge. Beispiel:
Wenn ein Kapitel in Abschnitte unterteilt wird, dann gibt es mindestens
zwei Abschnitte in dem Kapitel.

\section{Gliederung: Abschnitt}

Nach jedem Gliederungs-Element (Kapitel, Abschnitt, Unterabschnitt)
steht zunächst ein beschreibender Text. Erst dann wird das Dokument
in Unterabschnitte unterteilt.

\subsection{Gliederung: Unter-Abschnitt}

Hier wird auf den Abschnitt \ref{chap:markeeineskapitels} verwiesen.
