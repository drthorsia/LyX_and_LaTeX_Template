
\chapter{Tabellen}

Die Tabelle \ref{tab:komma} zeigt exemplarisch den Aufbau einer Tabelle.

\begin{longtable}[c]{|r|c|}
\caption{Messwertabelle verschiedener Unterstationen \label{tab:komma}}
\tabularnewline
\hline 
\textbf{Station} & \textbf{Messwert~(V)}\tabularnewline
\endfirsthead
\hline 
4711 & \hphantom{000}3,14\hphantom{0}\tabularnewline
\hline 
0815 & \phantom{000}2,74\hphantom{0}\tabularnewline
\hline 
0123456789 & \hphantom{000}2,345\tabularnewline
\hline 
0123456 & \hphantom{00}12,3\hphantom{00}\tabularnewline
\hline 
007 & 2345,3\hphantom{00}\tabularnewline
\hline 
\end{longtable}

Bei Tabellen sind insbesondere folgende Punkte wichtig:
\begin{itemize}
\item Zusammengehörige Zahlenwerte sind untereinander angeordnet, auch wenn
eine Anordnung nebeneinander Platzsparender ist. Untereinander angeordnete
Zahlenwerte sind leichter zu vergleichen.
\item Zahlenwerte sind untereinander am Komma ausgerichtet. Wenn die Zahl
kein Komma enthält, wird sie dort ausgerichtet, wo das Komma stehen
würde (Nach der letzten Ziffer).
\item Die Einheit steht entweder in der Spaltenüberschrift oder hinter jedem
Zahlenwert.
\item Die Tabellenüberschrift steht über der Tabelle, nicht unterhalb
\item Tabellen werden im Text erklärt.
\item Sehr große Tabellen mit Messwerten oder gerechneten Werten werden
häufig besser als Grafik dargestellt.
\end{itemize}

